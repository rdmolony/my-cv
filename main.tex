% ---------------------------------------------------------
% Header
% ------

\documentclass[a4paper,11pt]{article}

\usepackage[utf8x]{inputenc} % for character encoding
\usepackage{lmodern,textcomp} % to select font type
\usepackage[usenames,dvipsnames]{xcolor} % for colored links and divider
\usepackage[left=1cm, right=1cm, bottom=1cm, top=1cm]{geometry} % for adding a page margin
\usepackage{enumitem} % for \item lists
\usepackage{tabularx} % for sections
\usepackage{titlesec} % for section titles
\usepackage{ifthen} % to enable if else statements
\usepackage{fontawesome5} % for social icons
\usepackage[unicode, draft=false]{hyperref} % for hyperlinks

% ------
% Style
% ------

% Set the default font to sans-serif for the entire document
\renewcommand{\familydefault}{\sfdefault}

% Remove page numbers and headers/footers
\pagestyle{empty}

% Remove paragraph indentation throughout the document
\setlength{\parindent}{0pt}

% Set the background color
\definecolor{backgroundcolor}{RGB}{248,250,252}
\colorlet{body}{backgroundcolor}

\definecolor{textcolor}{RGB}{15,23,42}
\colorlet{text}{textcolor}

% Define a color for hyperlinks
% & configure hyperlinks to use the defined color and other properties
\definecolor{linkcolour}{RGB}{0,51,102}
\hypersetup{colorlinks,breaklinks,urlcolor=textcolor,linkcolor=textcolor}

% Define custom column types for tables:
% L = Left-aligned column with specified width
% C = Center-aligned column with specified width
% R = Right-aligned column with specified width
\newcolumntype{L}[1]{>{\raggedright\let\newline\\\arraybackslash\hspace{0pt}}m{#1}}
\newcolumntype{C}[1]{>{\centering\let\newline\\\arraybackslash\hspace{0pt}}m{#1}}
\newcolumntype{R}[1]{>{\raggedleft\let\newline\\\arraybackslash\hspace{0pt}}m{#1}}

% Format section headings: large font, left-aligned, with a horizontal rule below
% & set the spacing around section headings: left, above, below
\titleformat{\section}{\Large\raggedright}{}{0em}{}[\titlerule]
\titlespacing{\section}{0pt}{5pt}{5pt}

%----------
% Commands
%----------
\newcommand{\cventry}[5]{%
    \begin{tabularx}{\linewidth}{ L{0.65\linewidth} R{0.3\linewidth} }
      \textbf{#1, #2} & #3 \\[4pt]
      \multicolumn{2}{@{}X@{}}{%
        \begin{minipage}[t]{0.97\textwidth}
          \begin{itemize}[nosep, leftmargin=2em, itemsep=4pt]
            #4
          \end{itemize}
        \end{minipage}
      }
    \end{tabularx}
    \vspace{5pt}
}

% Define skill categories and references
\newcommand{\skillref}[1]{\textsuperscript{\textit{#1}}}


% ---------------------------------------------------------
% Document
% ---------
\begin{document}

%-------
% Title
%-------
\begin{tabularx}{\linewidth}{R{0.3\linewidth} C{0.3\linewidth} L{0.3\linewidth}}
&\huge{\textbf{Rowan Molony}}& \\[7.5pt]

\href{tel:+353862120533}{\raisebox{-0.05\height}\faMobile \ \underline{+353 86 212 0533}} &
\href{mailto:rowanmolony@gmail.com}{\raisebox{-0.05\height}\faEnvelope \ \underline{rowanmolony@gmail.com}} &
\href{https://linkedin.com/in/rowanmolony}{\raisebox{-0.05\height}\faLinkedin\ \underline{rowanmolony}} \\
\end{tabularx}

\vspace{10pt}

%--------
% Skills
%--------

\section{Skills}

\begin{minipage}[t]{0.97\textwidth}
  \begin{itemize}[nosep, leftmargin=2em, itemsep=4pt]
      \item \textbf{Backend}: Server-Side Rendered HTML (Django, HTMX)\skillref{2,4}, REST APIs (Django Rest Framework)\skillref{2,4}, GraphQL (AWS AppSync)\skillref{3}, Relational Databases (MySQL, Postgres, TimescaleDB)\skillref{2,4}, Task Queues (Dramatiq, Redis)\skillref{2,4}, Document Stores (AWS DynamoDB)\skillref{3}
      \item \textbf{Frontend}: Single Page Applications (React)\skillref{3}, CSS (TailwindCSS)\skillref{3}, Component Tests (StorybookJS)\skillref{3}
      \item \textbf{Good Practice}: Version Control (git)\skillref{1,2,3,4}, Code Collaboration (GitHub)\skillref{1,2,3,4}, Test-Driven Development (pytest\skillref{1,2,4}, Selenium\skillref{2,4}, nbdev\skillref{3}), Continuous Integration (GitHub Actions)\skillref{1,2,3,4}
      \item \textbf{Infrastructure}: Containerisation (Docker\skillref{1,2,3,4}, Nix\skillref{3}), Continuous Deployment (AWS Amplify)\skillref{3}, Infrastructure as Code (Terraform, AWS CDK)\skillref{3}
      \item \textbf{Data Analysis}: Linear Optimization (bespoke)\skillref{3}, Data Wrangling (Pandas)\skillref{1}, Workflow Orchestration (Ploomber, Prefect)\skillref{1}, Geospatial Analysis (GeoPandas)\skillref{1}, Machine Learning (scikit-learn)\skillref{1}, Network Analysis (networkx)\skillref{1}
  \end{itemize}
\end{minipage}

\vspace{10pt}

%------------
% Experience
%------------
\section{Experience}

\cventry
  {Software Engineer / Founder}
  {Thalora Ltd\skillref{4}}
  {Dec 2023 - Present}
  {
      \item 
      I formed Thalora Ltd through which I continued working with MRP on a consultancy basis.
      \vspace{5pt}
  }

\cventry
  {Software Engineer / Co-Founder}
  {PowerScope Ltd\skillref{3}}
  {Jul 2020 - Feb 2022}
  {
      \item
      Terry McGrenaghan \& I initially formed PowerScope Ltd to empower solar panel installers to model the investment case for battery storage for commercial projects, with the help of our software, without having to rely on external consultants.
      \item
      In so doing, I (a) built a linear optimisation framework (in Python) tailored for timeseries data \& a battery dispatch model on top of it, (c) a financial modelling framework \& model, (c) an interactive web application on the AWS Amplify stack.
  }

\cventry
  {Web Developer}
  {Mainstream Renewable Power (MRP)\skillref{2}}
  {Oct 2021 - Dec 2023}
  {
      \item
      Over several years I fully rebuilt the backend of the MRP data platform, StationManager, to resolve its reliability issues.
      \item
      In so doing, I (a) captured the developer environment, (b) grew a test suite (unit, integration, \& end to end) to cover common usage patterns in continuous integration, (c) simplified the system's architecture by removing, combining, or replacing its constituent components with more appropriate tools. The resulting system was far more robust, understandable and maintainable (including thousands of lines of code smaller) while still achieiving the same functionality.
  }

\cventry
  {Data Analyst}
  {Codema\skillref{1}}
  {Aug 2019 - Oct 2021}
  {
      \item 
      I tasked with modelling Dublin's built environment, as part of the Dublin Region Energy Masterplan, to enable estimating the impact of energy policy on Dublin's energy system.
      \item
      In so doing, I built my own toolkit and energy model in the open \href{https://github.com/codema-dev/projects}{\underline{on GitHub}} on top of open source libraries.
  }

%-----------
% Education
%-----------
\section{Education}

\cventry
  {BAI \& MAI in Mechanical Engineering}
  {Trinity College, Dublin}
  {Sept 2014 - June 2019}
  {
      \item 
      In 2018, I graduated my Bachelor's (BAI) with 1st class honours (gold medal).
      \item
      The degree remained multidisciplinary for two years, which exposed
      me to multiple courses in programming languages C++ \& Matlab.

      \item 
      In 2019, I completed a 1st class honours Masters (MAI), including a thesis in energy systems analysis.
      \item
      I was one of eight people selected to participate in a one year
      user-centered design project during my Masters year.
      My team of four were funded by a corporate sponsor to explore an open-ended design problem.
  }

\end{document}

% ---------------------------------------------------------